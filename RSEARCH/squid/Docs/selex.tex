\section{ SELEX alignment file format }

\subsection{ Example of a simple SELEX format file}

\begin{verbatim}
# Example selex file

seq1     ACGACGACGACG.
seq2     ..GGGAAAGG.GA
seq3     UUU..AAAUUU.A

seq1  ..ACG
seq2  AAGGG
seq3  AA...UUU
\end{verbatim}

SELEX is an interleaved multiple alignment format that evolved as an
intuitive format.  SELEX files are easy to write and manipulate
manually with a text editor.  It is usually easy to convert other
alignment formats into SELEX format; the output of the CLUSTALV
multiple alignment program and GCG's MSF format are similar
interleaved formats. Because it evolved to accomodate different user
input styles, it is very tolerant of various inconsistencies such as
different gap symbols, varying line lengths, etc.

As the format evolved, more features have been added. To maintain
compatibility with past alignment files, the new features are added
using a reserved comment style. These extra features are usually
maintained by automated SELEX-generating software, such as the {\tt
koala} sequence alignment editor or my {\tt cove} and {\tt hmm} sequence
analysis packages. This extra information includes consensus and
individual RNA or protein secondary structure, per-sequence weights, a
reference coordinate system for the columns, and database source
information including name, accession number, and coordinates (for
subsequences extracted from a longer source sequence).

\subsection {Specification of a SELEX file}

\begin{enumerate}
\item
Any line beginning with a \verb+#=+ as the first two characters is a
machine ``comment''.  \verb+#=+ comments are reserved for additional
data about the alignment. Usually these features are maintained by
software such as the {\tt koala} editor, not by hand.

\item
All other lines beginning with a \verb+%+ or \verb+#+ as the first
character is a user comment.  User comments are ignored by all
software. Any number of comments may be included.

\item
Lines of data consist of a name followed by a sequence. The total
length of the line must be smaller than 1024 characters.

\item
Names must be a single word. Any non-whitespace characters are
accepted.  No spaces are tolerated in names: names MUST be a
single word.

\item
In the sequence, any of the characters \verb+-_.+ or a space are
recognized as gaps. Gaps are converted to a '.'. Any other characters
are interpreted as sequence.  Sequence is case-sensitive. There is a
common assumption by my software that upper-case symbols are used for
consensus (match) positions and lower-case symbols are used for
inserts. This language of ``match'' versus ``insert'' comes from the
hidden Markov model formalism \cite{Krogh94}. To almost all of my
software, this isn't important, and it immediately converts the
sequence to all upper-case after it's read.

\item
Multiple different sequences are grouped in a block of data lines.
Blocks are separated by blank lines. No blank lines are tolerated
between the sequence lines in a block. Each block in a multi-block
file of a long alignment must have its sequences in the same order in
each block. The names are checked to verify that this is the case; if
not, only a warning is generated. (In manually constructed files, some
users may wish to use shorthand names after the first block with full
names, but this isn't recommended.)
\end{enumerate}

\subsection {Special comments}

\subsubsection {Secondary structure}

I use one-letter codes to indicate secondary structures. Secondary
structure strings are aligned to sequence blocks just like additional
sequences.

For RNA secondary structure, the symbols \verb+>+ and \verb+<+ are
used for base pairs (pairs point at each other).  \verb-+- indicate
other single-stranded positions, {\tt .} indicates unassigned bases.
This description follows \cite{Konings89}.  For protein secondary
structure, I use {\tt E} to indicate residues in $\beta$-sheet, {\tt
H} for those in $\alpha$-helix, {\tt L} for those in loops, and {\tt
.} for unassigned residues.

RNA pseudoknots are represented by alphabetic characters, with upper
case letters representing the 5' side of the helix and lower case
letters representing the 3' side. Note that this restricts the
annotation to a maximum of 26 pseudoknots per sequence.

Lines beginning with \verb+#=SS+ or \verb+#=CS+ are individual or
consensus secondary structure data, respectively.  \verb+#=SS+
individual secondary structure lines must immediately follow the
sequence they are associated with.  There can only be one \verb+#=SS+
per sequence. \verb+#=CS+ consensus secondary structure predictions
precede all the sequences in each block. There can only be one
\verb+#=CS+ per file.

\subsubsection {Reference coordinate system}

Alignments are usually numbered by some reference coordinate system,
often a canonical molecule. For instance, tRNA positions are numbered
by reference to the positions of yeast tRNA-Phe.

A line beginning with \verb+#=RF+ preceding the sequences in a block
gives a reference coordinate system. Any non-gap symbol in the
\verb+#=RF+ line indicates that sequence positions in its columns are
numbered. For instance, the \verb+#=RF+ lines for a tRNA alignment
would have 76 non-gap symbols for the canonical numbered columns; they
might be the aligned tRNA-Phe sequence itself, or they might be just
X's.

\subsubsection {Sequence header}

Additional per-sequence information can be placed in a header before
any blocks appear. These lines, one per sequence and in exactly the
same order as the sequences appear in the alignment, are formatted
like \verb+#=SQ <seqname> <weight> <database source name> <database
accession> <source coordinates as start..stop::original length>
<description>+.

This information includes a sequence weight (for compensating for
biased representation of subfamilies of sequences in the alignment);
source information, if the sequence came from a database, consisting
of identifier, accession number, and source coordinates; and a
description of the sequence.

If a \verb+#=SQ+ line is present, all the fields must be present.  If
no information is available for a field, use '-' for all the fields
except the source coordinates, which would be given as '0'.

\subsubsection {Author}

The first non-comment, non-blank line of the file may be a \verb+#=AU+
``author'' line. There is a programmatic interface for
alignment-generating programs to record a short comment like \verb+11
November 1993, by Feng-Doolittle v. 2.1.1+, and this comment will be
recorded on the \verb+#=AU+ line by \verb+WriteSELEX()+.



