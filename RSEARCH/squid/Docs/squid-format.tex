\newpage
\section {SQUID format}

SQUID format is a sequence database format similar to the PIR,
GenBank, and EMBL formats. The primary difference is that SQUID format
may optionally contain secondary structure annotation information for
the sequence. No other sequence format allows secondary structure
annotation, which is why SQUID format became necessary.

An example SQUID format file:

\begin{verbatim}
NAM  DY9990
SRC  HSTGYA M27547 76..169::196
DES  Human Tyr-tRNA gene, clone pM6.
SEQ  +SS
       1 ccttcgatagctcagctggtagagcggaggactgtagactgcggaaacgt
         >>>>>>>..>>>>........<<<<.>>>>>...................
      51 ttgtggacatccttaggtcgctggttcaattccggctcgaagga
         .........<<<<<.....>>>>>.......<<<<<<<<<<<<.
++
NAM  DY9991
SRC  HSTRNAYE M55611 1..93::93
DES  Human Tyr-tRNA precursor.
SEQ  +SS
       1 ccttcgatagctcagctggtagagcggaggactgtagcctgtagaaacat
         >>>>>>>..>>>>........<<<<.>>>>>...................
      51 ttgtggacatccttaggtcgctggttcgattccggctcgaagg
         .........<<<<<.....>>>>>.......<<<<<<<<<<<<
++
NAM  DA0260
SEQ
       1 GGGCGAAUAGUGUCAGCGGGAGCACACCAGACUUGCAAUCUGGUAGGGAG
      51 GGUUCGAGUCCCUCUUUGUCCACCA
++
\end{verbatim}


\subsection {Specification of a SQUID file}

\begin{enumerate}
\item There must be a line of the form \verb+NAM  <sequence name>+.

\item There may be an optional line \verb+SRC <id> <acc>
<start>..<stop>::<olen>+, which specified a database source for this
sequence, giving the database identifier (name), accession number,
start and end position in the database sequence, and the original
length of the database sequence, respectively.  If a \verb+SRC+ line
is present, all of these values must be specified.  If any values are
unknown, they may be set to \verb+-+ in the case of \verb+<id>+ and
\verb+<acc>+ and \verb+0+ in the case of \verb+<start>+, \verb+<stop>,
and \verb+<olen>+, and in these cases the values will be ignored.

\item There may be an optional line  \verb+DES <description>+ giving
a one-line description of the sequence.

\item There must be a line of the form \verb-SEQ +SS- or \verb-SEQ-.
If the line contains \verb-+SS-, it means that the record contains
secondary structure annotation interleaved with the sequence.

\item The sequence (and optional structure) immediately follow. There may be
optional numbering either before or after the sequence. The number of
characters per line is unimportant. Spaces and tabs are ignored.
There must be no non-numeric non-space characters on any lines except
sequence or structure annotation characters. Structure annotation is
fairly free-form; any alphabetic character or character in the set
\verb/_.-*?<>{}[]()!@#$%^&=+;:'|`~"\/ is accepted. There must
be one such character for every sequence character (preferably aligned
to the sequence, but in fact this is not checked for). Note that
spaces in the secondary structure annotation are not permitted,
except where they are aligned to gaps in the sequence.

\item Sequence records are separated by a line of the form \verb-++-.
\end{enumerate}



   


